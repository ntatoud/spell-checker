\documentclass{article}
\usepackage[utf8]{inputenc}
\usepackage{enumitem}
\usepackage{amssymb}
\usepackage[hmargin=2cm,vmargin=0cm]{geometry}
\begin{document}
    \noindent
    \textbf{Nom}: Mot \\
    \textbf{Utilise}: Chaine de caracteres,NaturelNonNul,Caractere,Booleen \\
    \textbf{Opérations}: \begin{itemize}[label=$\ $, leftmargin=2cm, itemsep=0cm]
        \item estUnMotValide: Chaine de caracteres $\rightarrow $ Booleen
        \item creerUnmot: ChaineDeCaracteres $ \rightarrow$ Mot
        \item longueur: Mot $ \rightarrow$  NaturelNonNul
        \item accederAuIemeCaractere: Mot $ \times $ NaturelNonNul $ \nrightarrow$  Caractere
        \item sontIdentiques: Mot $ \times $ Mot $ \rightarrow$  Booléen
        \item fixerIemeCaractere:  Mot $ \times $ NaturelNonNul $ \times $ Caractere $ \rightarrow$ Mot
        \item fixerLongueur: Mot $ \times $ Naturel $ \rightarrow$ Mot
    \end{itemize}
    
    \textbf{Sémantique}: \begin{itemize}[label=$\- $, leftmargin=2cm, itemsep=0cm]
      \item creerUnMot:création d’un mot à partir d’une chaine de caractère.
        \item estUnMotValide: verifier si le mot est bien composé des lettres, renvoit VRAI si le mot contient que les lettres et renvoit FAUX sinon.
        \item longueur: donner la longueur d’un mot.
        \item accederAuIemeCaractere: accéder à un certain caractère d'un mot en precisant son indice.
         \item sontIdentiques: vérifier si deux mots sont identiques.
         \item fixerIemeCaractere: permet de changer une lettre du mot à une position donnée. 
         \item fixerLongueur: permet de fixer la longueur d'un mot.
    \end{itemize}

    \textbf{Préconditions}: \begin{itemize}[label=$\- $, leftmargin=2cm, itemsep=0cm]
        \item accederAuIemeCaractere(mot, i) : 1 $\leq$ i $\leq$ longueur(mot)
    \end{itemize}
\end{document}
