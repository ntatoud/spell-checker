\documentclass{article}
\usepackage[utf8]{inputenc}
\usepackage{enumitem}
\usepackage{amssymb}
\usepackage[hmargin=2cm,vmargin=0cm]{geometry}
\begin{document}
    \noindent
    \textbf{Nom}: Mot \\
    \textbf{Utilise}: Chaine de caracteres,NaturelNonNul,Caractere,Booleen \\
    \textbf{Opérations}: \begin{itemize}[label=$\ $, leftmargin=2cm, itemsep=0cm]
        \item estUnMotValide: Chaine de caracteres $\rightarrow $ Booleen
        \item creerUnmot: ChaineDeCaracteres $ \rightarrow$ Mot
        \item longueur: Mot $ \rightarrow$  NaturelNonNul
        \item accederAuIemeCaractere: Mot $ \times $ NaturelNonNul $ \nrightarrow$  Caractere
%         \item sontIdentiques Mot $ \times $ Mot $ \rightarrow$  Booléen
        \item remplacerIemeLettre : Mot $\times$ NaturelNonNul $\nrightarrow$ Ensemble\textless Mot\textgreater
        \item supprimerIemeLettre : Mot $\times$ NaturelNonNul $\nrightarrow$ Mot
        \item inverserDeuxLettresConsecutives : Mot $\times$ NaturelNonNul $\nrightarrow$ Mot
        \item insererLettre : Mot $\times$ NaturelNonNul $\nrightarrow$ Ensemble\textless Mot\textgreater
        \item decomposerMot : Mot $\times$ NaturelNonNul $\nrightarrow$ Mot $\times$ Mot
        \item reduireLaCasse : Mot $\rightarrow$ Mot
    \end{itemize}
    
    \textbf{Sémantique}: \begin{itemize}[label=$\- $, leftmargin=2cm, itemsep=0cm]
      \item creerUnMot:création d’un mot à partir d’une chaine de caractère.
        \item estUnMotValide: verifier si le mot est bien composé des lettres, renvoit VRAI si le mot contient que les lettres et renvoit FAUX sinon.
        \item longueur: donner la longueur d’un mot.
        \item accederAuIemeCaractere: accéder à un certain caractère d`un mot en precisant son indice.
         \item sontIdentiques : vérifier si deux mots sont identiques.
                 \item remplacerIemeLettre : Remplace la ième lettre du mot par les 25 autres lettres de l'alphabet tour à tour
        \item supprimerIemeLettre : Supprime la ième lettre du mot
        \item inverserDeuxLettresConsecutives : Inverse la lettre i et la lettre i+1
        \item insererLettre : Insère les 26 lettres de l'alphabet tour à tour entre la lettre i et la lettre i+1
        \item decomposerMot : Sépare le mot en deux parties, de part et d'autre de la lettre i
        \item reduireLaCasse : Change tous les caractères majuscules en minuscule
    \end{itemize}

    \textbf{Préconditions}: \begin{itemize}[label=$\- $, leftmargin=2cm, itemsep=0cm]
        \item accederAuIemeCaractere(mot, i) : 1 $\leq$ i $\leq$ longueur(mot)
                \item remplacerIemeLettre(mot, i) : i $\leq$ longueur(mot)
        \item supprimerIemeLettre(mot, i) : longeur(mot) $>$ 1 et i $\leq$ longueur(mot)
        \item inverserDeuxLettresConsecutives(mot, i) : longeur(mot) $>$ 1 et i $\leq$ longueur(mot)
        \item insererLettre(mot, i) : i $\leq$ longueur(mot)+1
        \item decomposerMot(mot, i) : i $\leq$ longueur(mot)
    \end{itemize}
\end{document}
