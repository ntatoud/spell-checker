\documentclass{article}
\usepackage[utf8]{inputenc}
\usepackage{enumitem}
\usepackage{amssymb}
\usepackage[hmargin=2cm,vmargin=0cm]{geometry}
\begin{document}
    \noindent
    \textbf{Type Mode} = \{lecture,ecriture\} \\
    \textbf{Nom}: FichierTexte \\
    \textbf{Utilise}: \textbf{Chaine de caracteres,Mode,Caractere,Booleen} \\
    \textbf{Opérations}: \begin{itemize}[label=$\ $, leftmargin=2cm, itemsep=0cm]
        \item fichierTexte: \textbf{Chaine de caracteres $\rightarrow $ FichierTexte}
        \item ouvrir: \textbf{FichierTexte $ \times $ Mode $ \nrightarrow$ Fichier}
        \item fermer: \textbf{FichierTexte$  \nrightarrow $ FichierTexte}
        \item estOuvert: \textbf{FichierTexte$ \nrightarrow$  Booleen}
        \item mode: \textbf{FichierTexte $\nrightarrow $ Mode}
        \item finFichier: \textbf{FichierTexte $ \nrightarrow $ Booleen}
        \item ecrireChaine: \textbf{FichierTexte $ \times $ Chaine $ \nrightarrow$  FichierTexte}
        \item lireChaine: \textbf{FichierTexte $ \nrightarrow$  FichierTexte $ \times $ Chaine}
        \item ecrireCaractere: \textbf{FichierTexte $ \times $ Caractere $ \nrightarrow$  FichierTexte}
        \item lireCaractere: \textbf{FichierTexte $ \nrightarrow$  FichierTexte $ \times $ Caractere}
    \end{itemize}
    
    \textbf{Sémantique}: \begin{itemize}[label=$\- $, leftmargin=2cm, itemsep=0cm]
        \item fichierTexte: création d’un fichier texte à partir d’un fichier identifié par son nom.
        \item ouvrir: ouvre un fichier texte en lecture ou écriture. Si le mode est écriture et que le fichier existe, alors ce dernier est écrasé.
        \item fermer: ferme un fichier texte.
        \item lireCaractere: lit un caractère à partir de la position courante du fichier.
        \item lireChaine: lit une chaîne (jusqu'à un retour à la ligne ou la fin de fichier) à partir de la position courante du fichier.
        \item ecrireCaractere: écrit un caractère à partir de la position courante du fichier.
        \item ecrireChaine: écrit une chaîne suivie d'un retour à la ligne à partir de la position courante du fichier.
    \end{itemize}

    \textbf{Préconditions}: \begin{itemize}[label=$\- $, leftmargin=2cm, itemsep=0cm]
        \item ouvrir(f): non(estOuvert(f))
        \item fermer(f): estOuvert(f)
        \item finFichier(f): mode(f)=lecture
        \item lireXX(f): estOuvert(f) et mode(f)=lecture et non finFichier(f)
        \item ecrireXX(f): estOuvert(f) et mode(f)=ecriture
    \end{itemize}
\end{document}
