\documentclass{article}
\usepackage[utf8]{inputenc}
\usepackage{enumitem}
\usepackage{amssymb}
\usepackage[hmargin=2cm,vmargin=0cm]{geometry}
\begin{document}
    \noindent
    Nom: Dictionnaire \\
    Utilise: Mot,Mode,FichierTexte,Booleen \\
    Type Mode={Lecture,Ecriture} \\
    Opérations: \begin{itemize}[label=$\ $, leftmargin=2cm, itemsep=0cm]
        \item creerDictionnaire: FichierTexte $\rightarrow $ Dictionnaire
        \item contientMot: Dictionnaire $\times$ Mot $\rightarrow $ Booléen
        \item ajouterListeMots : Dictionnaire $\times$ FichierTexte $ \rightarrow$ Dictionnaire
        \item ouvrirDictionnaire : Dictionnaire $\times$ Mode $ \rightarrow$ Dictionnaire
        \item fermerDictionnaire : Dictionnaire $ \nrightarrow$ Dictionnaire
        \item estOuvert : Dictionnaire $ \rightarrow$ Booleen
        \item modeDictionnaire : Dictionnaire $ \nrightarrow$ Mode
        
        
    \end{itemize}
    
    Sémantique: \begin{itemize}[label=$\- $, leftmargin=2cm, itemsep=0cm]
        \item creerDictionnaire : création d’un dictionnaire à partir d’un fichierTexte (trie et crée l'arborescence)
        \item contientMot : renvoit VRAI si le mot est dans le dictionnaire, FAUX sinon
        \item ajouterListeMots: ajoute une liste de mots dans le dictionnaire,en les insérant dans l'aborescence, en écrasant si le mot est déjà présent
        \item ouvrirDictionnaire : ouvre le dictionnaire dans le mode choisi
        \item fermerDictionnaire : referme le dictionnaire ouvert
        \item estOuvert : renvoie le booléen correspondant à l'état ouvert ou fermé du dictionnaire
        \item modeDictionnaire : renvoie le mode du dictionnaire : Lecture ou Ecriture
    \end{itemize}

    Préconditions: \begin{itemize}[label=$\- $, leftmargin=2cm, itemsep=0cm]
        \item fermerDictionnaire : estOuvert(Dictionnaire)
        \item modeDictionnaire : estOuvert(Dictionnaire)
    \end{itemize}
\end{document}
