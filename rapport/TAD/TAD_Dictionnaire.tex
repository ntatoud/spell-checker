\documentclass{article}
\usepackage[utf8]{inputenc}
\usepackage{enumitem}
\usepackage{amssymb}
\usepackage[hmargin=2cm,vmargin=0cm]{geometry}
\begin{document}
    \noindent
    \\
    \textbf{Nom}: Dictionnaire\\
    \textbf{Type} Dictionnaire = ArbreDeLettres \\
    \textbf{Utilise} : Mot, FichierTexte, Ensemble$<$Mot$>$,Booléen\\
    \textbf{Opérations:} \begin{itemize}[label=$\ $, leftmargin=2cm, itemsep=0cm]
        \item genererArbreAvecEnsembleDeMot: \textbf{Ensemble$<$Mot$>$ $\nrightarrow $ Dictionnaire}
        \item estUnMotDuDictionnaire: \textbf{Dictionnaire $\times$ Mot $\rightarrow $ Booléen}
        \item chargerDico : \textbf{FichierTexte$ \rightarrow$ Dictionnaire}
        \item sauvegarderDico: \textbf{Dictionnaire$\rightarrow$ FichierTexte}
        
    \end{itemize}
    \textbf{Préconditions} :
    \begin{itemize}[label=$\ $, leftmargin=2cm, itemsep=0cm]
     	\item genererArbreAvecEnsembleDeMot(lesMots) : non estVide(lesMots)
    \end{itemize}

    \textbf{Sémantique: }\begin{itemize}[label=$\- $, leftmargin=2cm, itemsep=0cm]
        \item genererArbreAvecEnsembleDeMot : création d'un arbre représentant notre dictionnaire à l'aide d'un ensemble de mots
        \item estUnMotDuDictionnaire : renvoie VRAI si le mot est dans le dictionnaire, FAUX sinon
        \item chargerDico: recrée le dictionnaire sous forme d'arbre correspondant au fichier texte sauvegardé
        \item sauvegarderDico : enregistre l'arbre sous forme de fichier texte
    \end{itemize}
\end{document}
