\documentclass{article}
\usepackage[utf8]{inputenc}
\usepackage{enumitem}
\usepackage{amssymb}
\usepackage[hmargin=2cm,vmargin=0cm]{geometry}
\begin{document}
    \noindent
    \\
    \textbf{Nom}: CorrecteurOrthographique \\
    \textbf{Utilise}: \textbf{Mot, Dictionnaire, Ensemble$<$Mots$>$} \\
    \textbf{Opérations}: \begin{itemize}[label=$\ $, leftmargin=2cm, itemsep=0cm]
        \item correcteur : \textbf{Dictionnaire $\times$ Mot $\rightarrow$ CorrecteurOrthographique}
        \item obtenirMotACorriger : \textbf{CorrecteurOrthographique $\rightarrow$ Mot}
        \item obtenirDictionnaire : \textbf{CorrecteurOrthographique $\rightarrow$ Dictionnaire}
        \item obtenirCorrections : \textbf{CorrecteurOrthographique $\rightarrow$ Ensemble$<$Mots$>$}
        \item fixerDico : \textbf{CorrecteurOrthographique $\times$ Dictionnaire $\rightarrow$ CorrecteurOrthographique}
        \item fixerMotACorriger : \textbf{CorrecteurOrthographique $\times$ Mot $\rightarrow$ CorrecteurOrthographique}
        \item ajouterNouvellesCorrections : \\ \textbf{CorrecteurOrthographique $\times$ Ensemble$<$Mot$>$ $\rightarrow$ CorrecteurOrthographique}
        \item trouverCorrectionsPossibles : \textbf{CorrecteurOrthographique $\rightarrow$ CorrecteurOrthographique}
        \item remplacerIemeLettreEnBoucle : \textbf{Mot $\times$ Naturel $\rightarrow$ Ensemble$<$Mot$>$}
        \item strategieRemplacerLettres : \textbf{CorrecteurOrthographique $\rightarrow$ CorrecteurOrthographique}
        \item strategieSupprimerLettres : \textbf{CorrecteurOrthographique $\rightarrow$ CorrecteurOrthographique}
        \item strategieInverserDeuxLettresConsecutives : \textbf{CorrecteurOrthographique $\rightarrow$ CorrecteurOrthographique}
        \item insererIemeLettreEnBoucle : \textbf{Mot $\times$ Naturel $\rightarrow$ Ensemble$<$Mot$>$}
        \item strategieInsererLettres : \textbf{CorrecteurOrthographique $\rightarrow$ CorrecteurOrthographique}
        \item strategieDecomposerMot : \textbf{CorrecteurOrthographique $\rightarrow$ CorrecteurOrthographique}
    \end{itemize}
    \textbf{Sémantique}: \begin{itemize}[label=$\- $, leftmargin=2cm, itemsep=0cm]
        \item obtenirMotACorriger : Permet d'accèder au mot à corriger 
        \item obtenirDictionnaire : Permet d'accèder au dictionnaire 
        \item obtenirCorrections : Permet d'accèder aux corrections du mot
        \item fixerDico : Donne un dictionnaire à utiliser au correcteur.
        \item fixerMotACorriger : Donne un mot à corriger au correcteur.
        \item ajouterNouvellesCorrections : Ajoute de nouvelles corrections du mot à corriger au correcteur.
        \item trouverCorrectionsPossibles : Renvoie l'ensemble des corrections possibles du mot à corriger.
        \item remplacerIemeLettreEnBoucle : Remplace une lettre du mot par toutes les autres de l'alphabet, une par une
        \item strategieRemplacerLettres : Remplace toutes les lettres du mot par tous les caractères de l'alphabet tour à tour et ajoute les corrections valides au correcteur
        \item strategieSupprimerLettres : Supprime les lettres du mot tour à tour et ajoute les corrections valides au correcteur
        \item strategieInverserDeuxLettresConsecutives : Inverse les lettres du mot deux à deux, les unes après les autres et ajoute les corrections valides au correcteur
        \item remplacerIemeLettreEnBoucle : Insère toutes les lettres de l'alphabet une par une à un endroit du mot
        \item strategieInsererLettres : Insère un par un tous les caractères alphabétiques à tous les endroits du mot et ajoute les corrections valides au correcteur
        \item strategieDecomposerMot : Décompose le mot en deux parties de toutes les façons possibles et ajoute les corrections valides au correcteur
    \end{itemize}
        
    \textbf{Préconditions}: \begin{itemize}[label=$\- $, leftmargin=2cm, itemsep=0cm]
        \item correcteur(unDico, unMotFaux) : non(estUnMotDuDictionnaire(unDico, unMotFaux)
        \item fixerMotACorriger(unCorrecteur, unMotFaux) : \\non(estUnMotDuDictionnaire(obtenirDictionnaire(unCorrecteur), unMotFaux))
    \end{itemize}
        
\end{document}
