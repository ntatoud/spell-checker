\documentclass{article}
\usepackage{array}
\usepackage{classeRapport}
\usepackage{pdfpages}

\begin{document}

\PageDeGarde	
{rien.png} % image sur la page de garde
{Correcteur Orthographique} % titre principal
{Sous-titre} % sous-titre
{Fatiha \textsc{Hammouche} \\
Florine \textsc{Chevrier} \\
Loïck \textsc{Toupin} \\
Noé \textsc{Tatoud}}
{Algo - ITI - 2021} % bas de page

\Page{INSALogo}{rien.png} % logo de bas de page (en bas a droite)


\tableofcontents

\clearpage
\section{Introduction}
Dans le cadre de nos études dans la filière ITI à l’INSA de Rouen, nous avons réalisé un projet d’algorithmie en C. Le but de ce projet était de réaliser un correcteur orthographique. Ce projet est le premier que nous avons eu à réaliser du début à la fin en autonomie presque complète. Cela nous a permis de faire face à de nombreuses difficultés et ainsi de progresser dans de divers domaines. \\
En effet, ce projet a évidemment sollicité nos connaissances algorithmiques mais également notre capacité à travailler en groupe. Nous avons appris à nous organiser, mais aussi à mieux communiquer. Nous avons donc appris à nous adapter aux autres, et surtout à coder de façon claire et précise pour que nos collègues puissent comprendre ce que nous avions fait. Afin de faciliter la gestion de ce projet, nous devions utiliser Git dont nous nous étions déjà servi à d’autres occasions mais pour la plupart d’entre nous, nous ne le maîtrisions pas encore. L’utilisation de cette plateforme est donc une autre compétence essentielle que nous avons pu développer. Nous avons tous aussi progressé en C, qui est un langage que nous avons commencé à étudier au début de l’année, ainsi qu’à documenter notre code avec Doxygen. Nous avons également amélioré nos compétences en \LaTeX que nous avons utilisé pour la rédaction du rapport. \\
Nous présentons donc dans ce rapport le résultat de notre travail, en commençant par l'analyse, puis la conception préliminaire et enfin la conception détaillée.

\clearpage
\section{TAD}
	\includepdf[pages=-]{PDF/tad.pdf}


\clearpage
\section{Conception des TAD}
	\includepdf[pages=-]{PDF/conception.pdf}

\end{document}
