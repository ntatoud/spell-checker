\documentclass{article}
\usepackage[utf8]{inputenc}
\usepackage{enumitem}
\usepackage{amssymb}
\usepackage{times}
\usepackage{fancyhdr,graphicx,amsmath,amssymb}
\usepackage[ruled,vlined, french]{algorithm2e}
    \SetKwInput{Declaration}{Declaration}
    \SetKw{KwA}{a}
    \SetKw{Retour}{retourner}
    \SetKwBlock{Deb}{debut}{fin}
    \SetKwIF{Si}{SinonSi}{Sinon}{si}{alors}{sinon si}{sinon}{finsi}
    \SetKwFor{Tq}{tant que}{faire}{finTantQue}
    \SetKwFor{Pour}{pour}{faire}{finPour}
    \SetKwRepeat{Repeter}{repeter}{jusqu’a}
    \SetKw{Pre}{Precondition(s)}
\usepackage[hmargin=2cm,vmargin=0cm]{geometry}

\begin{document}
    \pagestyle{empty}
    \noindent
    \\
    On suppose possèder la conception des TAD Liste, ainsi que de la SDD ListeChainee et donc toutes les opérations rattachées
    \\
    \subsubsection*{Structures}
    \textbf{Type} ListeChainee = \^ \- Noeud \\
    \textbf{Type} Noeud = \textbf{Structure}
	\begin{itemize}[label=$\ $, leftmargin=2cm]
		 \item element : Element
		 \item listeSuivante: ListeChainee
	\end{itemize}
    \textbf{finStructure}
    \\
    \\
    \textbf{Type} Liste = \textbf{Structure}
	\begin{itemize}[label=$\ $, leftmargin=2cm]
		 \item lesElements : ListeChainee
		 \item nbElements : Entier
    \end{itemize}
    \textbf{finStructure} \\
    \\
    \\
    \textbf{Type} ArbreDeLettres = ArbreBinaire$<$Element$>$ \\
    \textbf{Type} Element = \textbf{Structure}
	\begin{itemize}[label=$\ $, leftmargin=2cm]
		 \item lettre : Caractere
		 \item fd,fg : ArbreDeLettres
		 \item estEnFinDeMot : Booléen
	\end{itemize}
    \textbf{finStructure}\\
    Accompagnées des fonctions obtenirLettre(a : ArbreDeLettres) et obtenirFils(a : ArbreDeLettres)
    
    
    \subsubsection*{Signatures}

	\begin{itemize}[label=$\ $, leftmargin=1cm]
		 \item \textbf{fonction} obtenirLettresFils(unArbre : ArbreDeLettres):Liste$<$Caractere$>$
         \item \textbf{fonction} obtenirFils(unArbre : ArbreDeLettres):Liste$<$ArbreDeLettres$>$
         \item \textbf{procédure} insererLettreALaBonnePlace(E/S unArbre : ArbreDeLettres, E indiceInsertion : NaturelNonNul, uneLettre : LettreDUnMot)
		 \item \textbf{procédure} insererMotALaBonnePlace(E/S unArbre : ArbreDeLettres, E unMot : Mot)
		 
         \item \textbf{fonction} genererArbreAvecEnsembleDeMot(lesMots : Ensemble$<$Mot$>$): ArbreDeLettres
         
        \begin{itemize}[label=$| $]
            \item \textbf{précondition:} non estOuvert(f)
         \end{itemize}
	\end{itemize} 
\end{document}
